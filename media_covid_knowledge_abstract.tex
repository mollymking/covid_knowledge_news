%----------------------------------------------------------------------------------------
%	PACKAGES AND OTHER DOCUMENT CONFIGURATIONS
%----------------------------------------------------------------------------------------
%!TEX root = media_covid_knowledge_abstract.tex

\PassOptionsToPackage{unicode=true}{hyperref} % options for packages loaded elsewhere
\PassOptionsToPackage{hyphens}{url}
%
\documentclass[11pt]{article}
\usepackage{appendix}
\usepackage{lmodern}
\usepackage{amssymb,amsmath}
\usepackage{ifxetex,ifluatex}
%-------%
% Journal submission specifications - ASA format
%\usepackage{mathptmx}
%\usepackage[,bottom=1in,top=1in,left=1in,right=1in]{geometry}
\usepackage{geometry}
%\usepackage[doublespacing]{setspace} % alternate way to add double space
%\usepackage[document]{ragged2e} % do not right-justify text
%\usepackage{enotez}
%  \let\footnote=\endnote

%Running header
\usepackage{fancyhdr}
\pagestyle{fancy}
\fancyhf{}% Clear header/footer
\fancyhead[L]{Paying Attention to the Pandemic}
\cfoot{\thepage} %page numbers bottom center

% where figures/tables in document
%\usepackage[nolists,tablesfirst]{endfloat} % forces all floats to appear at end of document
\usepackage[section]{placeins} %keeps floats in place, using command \FloatBarrier
%\usepackage[nolists, tabhead, fighead]{endfloat}
%\usepackage{flafter} %force floats to appear after they are defined
%-------%

 % packages for figures
\usepackage{array}  % tables
\usepackage{wrapfig}  % wrap text around narrow tables or figures
\usepackage{graphicx}  % for inserting graphics from file
\usepackage{blindtext}
\usepackage{asymptote}
\usepackage{subcaption}
\usepackage{rotating} %landscape tables
\usepackage{geometry}
\usepackage{pdflscape}

% for figures in multiple parts
\usepackage{caption}
%\DeclareCaptionLabelFormat{cont}{#1~#2\alph{ContinuedFloat}}
%\captionsetup[ContinuedFloat]{labelformat=cont}

%Bibliography
\usepackage{natbib}
\setcitestyle{round,aysep={},yysep={,},notesep={:}}

\ifnum 0\ifxetex 1\fi\ifluatex 1\fi=0 % if pdftex
  \usepackage[T1]{fontenc}
  \usepackage[utf8]{inputenc}
  \usepackage{textcomp} % provides euro and other symbols
\else % if luatex or xelatex
  \usepackage{unicode-math}
  \defaultfontfeatures{Ligatures=TeX,Scale=MatchLowercase}
\fi
% use upquote if available, for straight quotes in verbatim environments
\IfFileExists{upquote.sty}{\usepackage{upquote}}{}
% use microtype if available
\IfFileExists{microtype.sty}{%
\usepackage[]{microtype}
\UseMicrotypeSet[protrusion]{basicmath} % disable protrusion for tt fonts
}{}
\IfFileExists{parskip.sty}{%
\usepackage{parskip}
}{% else
\setlength{\parindent}{0pt}
\setlength{\parskip}{6pt plus 2pt minus 1pt}
}
\usepackage{hyperref}
\hypersetup{
            pdftitle={Knowledge of COVID-19 and News Sources},
            pdfborder={0 0 0},
            breaklinks=true}
\urlstyle{same}  % don't use monospace font for urls
\usepackage{graphicx,grffile}
\makeatletter
\def\maxwidth{\ifdim\Gin@nat@width>\linewidth\linewidth\else\Gin@nat@width\fi}
\def\maxheight{\ifdim\Gin@nat@height>\textheight\textheight\else\Gin@nat@height\fi}
\makeatother
% Scale images if necessary, so that they will not overflow the page
% margins by default, and it is still possible to overwrite the defaults
% using explicit options in \includegraphics[width, height, ...]{}
\setkeys{Gin}{width=\maxwidth,height=\maxheight,keepaspectratio}
\setlength{\emergencystretch}{3em}  % prevent overfull lines
\providecommand{\tightlist}{%
  \setlength{\itemsep}{0pt}\setlength{\parskip}{0pt}}
\setcounter{secnumdepth}{0}
% Redefines (sub)paragraphs to behave more like sections
\ifx\paragraph\undefined\else
\let\oldparagraph\paragraph
\renewcommand{\paragraph}[1]{\oldparagraph{#1}\mbox{}}
\fi
\ifx\subparagraph\undefined\else
\let\oldsubparagraph\subparagraph
\renewcommand{\subparagraph}[1]{\oldsubparagraph{#1}\mbox{}}
\fi

% set default figure placement to htbp
\makeatletter
\def\fps@figure{htbp}
\makeatother

%----------------------------------------------------------------------------------------
%	DOCUMENT CONTENT
%----------------------------------------------------------------------------------------

\begin{document}

\title{Paying Attention to the Pandemic: Knowledge of COVID-19 by News Sources and Demographics (Tentative Title)
\footnote{Corresponding Author: Molly M. King,
          Email: mmking@scu.edu.}}

\author{Molly M. King \\ Department of Sociology, Santa Clara University}
\date{}

\clearpage\maketitle

%----------------------------------------------------------------------------------------
\hypertarget{abstract}{%
\section{Abstract}\label{sec:abstract}}
%----------------------------------------------------------------------------------------
\pagestyle{fancy} %Running header
\setcounter{page}{1} %resets page number to 1


Never before has so much information been so immediately accessible to
so many. Yet the challenges in sorting through this information are perhaps
greater than ever. Previous research has looked at the role of social media and
other news sources on shaping the U.S. population’s understanding of the COVID-19
pandemic. However, what has not been studied is how this knowledge acquisition
is structured by the demographic characteristics of gender, race and ethnicity, and
income. Furthermore, how does uncertainty about this knowledge also differ by
demographic group membership? This study reveals how the use of
different news sources differentially shape access to accurate information about
COVID-19 and related topics for different demographics. I answer these questions
by analyzing recent Pew Research survey data asking respondents about their news
media consumption and their knowledge of COVID-19 science and related current
events information. I determine the effect of demographic group memberships in
shaping (mis)information and (un)certainty about COVID-19 received through news sources.



%----------------------------------------------------------------------------------------
\hypertarget{methods}{%
\section{Methods}\label{sec:methods}}
%----------------------------------------------------------------------------------------


In this analysis, I investigate factual knowledge and certainty using the Pew Pathways June 2020 American Trends Panel Wave 68 Survey. The survey features four questions of particular interest, which ask about factual knowledge about the COVID-19 pandemic and the economy during this time:

\begin{enumerate}
\def\labelenumi{(\arabic{enumi})}
  \item ``Is the national unemployment rate as reported by the government currently…'' (Correct answer is 15\%)
  \item ``As far as you know, are antibody tests for the coronavirus (also known as serology tests) intended to detect…'' (Correct answer is ``A previous infection'')
  \item ``Do you happen to know who Anthony Fauci is?'' (Correct answer is ``An infectious disease expert and government health adviser.'')
  \item ``As far as you know, how did states in the U.S. respond during the coronavirus outbreak?'' (Correct answer is ``Some states in the U.S. have not had a statewide stay-at-home order.'')
 \end{enumerate}
%

 %-------%
\hypertarget{model}{%
\subsection{Model}\label{sec:model}}

I use a two-step model to analyze the
levels of uncertain and correct answers for the three factual knowledge
question in this study (Figure~\ref{fig:TwoStepModelOfKnowledge}).

\begin{figure}[ht]
  \begin{center}
    \includegraphics{"~/Documents/SocResearch/Dissertation/theoretical_figures_and_images/two-step-model-of-knowledge"}
  \end{center}
  \caption[A model of responses to knowledge questions]
  {\emph{A model of responses to knowledge questions.}
   My analytical approach conceives of respondents following a two-step decision
   process when answering factual knowledge questions in surveys. First, the
   respondent decides if they are sufficiently certain to answer the question or
   not (a measure of collective certainty when aggregated). Others answer the
   question either `don't know' or some variation thereof. Conditional on
   answering, the respondent proceeds to step two, either getting the question
   correct or incorrect (a measure of misinformation).}
  \label{fig:TwoStepModelOfKnowledge}
\end{figure}

I separate my model of knowledge into two steps (see
Figure~\ref{fig:TwoStepModelOfKnowledge}). First, I analyze the proportion of
respondents who are uncertain about each factual knowledge
question by domain. To do this, I estimate the weighted population mean for the
response of `don't know' (or similar answers such as `not sure,' `no opinion,'
`prefer not to say,' and `don't know / refused') for each knowledge question
where such an answer was available to the respondent:

 $$ mean_{DK} = \frac{P(y = 1)}{N}. $$

In this model, \emph{y = 1} is the outcome variable (`don't know' answer), and
\emph{N} is the size of the dataset. The mean value `don't know' for the population for a single knowledge
question ($mean_DK$) is equal to the survey-weighted probability of the
respondent answering `don't know' (or a variant) divided by the total number of
survey respondents who were asked the question. Respondents who refused to answer are excluded from the analysis.

Second, I analyze the proportion of the population (out of those providing an answer related to the substance of the
question) who answered correctly and incorrectly. To do this, I estimate the
population mean (and standard error) for the binary variable indicating correct
answer:

 $$ mean_C = \frac{P(y = 1)}{N}. $$

In this model, \emph{N} is the size of the dataset, and \emph{y = 1} the outcome
variable (correct answer). The probability of answering correctly ($P(y = 1)$)
is calculated out of all individuals who provided a correct or incorrect answer and those who respond
`don't know'. Respondents who refused to answer are excluded from the analysis.
Hence, the mean value correct for the population for a single knowledge
question ($mean_C$) is equal to the survey-weighted probability of the
respondent getting the knowledge question correct divided by the total number of
survey respondents.


%----------------------------------------------------------------------------------------
\hypertarget{results}{%
\section{Preliminary Results}\label{sec:results}}
%----------------------------------------------------------------------------------------

Preliminary regression results find that both news source and demographic identity are significant predictors of whether or not people
answer questions related to COVID-19 and the pandemic economy correctly. Preliminary results showing odds ratios for each of these factors predicting correct knowledge of four knowledge questions are presented in Table~\ref{table:COVIDKnowledgeORs}. Race has the most consistent and dramatic effect on correct knowledge about antibody tests, even controlling for income and news source. In contrast, news source and income seem to have the most influence on whether respondents knew that Anthony Fauci is an infectious disease expert and government advisor. Income, predictably, had the most influence on whether someone answered the question about the national unemployment rate correctly -- although race was also a good predictor of knowing this fact.

Additional analyses carried out for the full paper will look at interactions of race and news source and of income and news source to see if different groups get different factual knowledge outcomes from consuming different news sources. Analyses will also investigate the role of gender, race and ethnicity, income, and education, in order to explore the role of intersectional identities (to the degree possible allowed by statistical power) on news consumption during the pandemic and knowledge of these questions. Additionally, in the full paper, I will transform categorical income into continuous income in order to make the coefficients more interpretable.


%----------------------------------------------------------------------------------------
\hypertarget{discussion}{%
\section{Preliminary Discussion}\label{sec:discussion}}
%----------------------------------------------------------------------------------------


This paper will describe a snapshot of COVID-19 and current events knowledge in
the U.S. as a function of demographic characteristics and news consumption. This
makes it a contribution in the style of Mannheim's sociology of knowledge
\citep{Mannheim, Swidler1994} and the study of demographic inequalities in
gender, class, and race and ethnicity. These are all important implications of
knowledge inequality and important reasons compelling its relevance to scholars
of inequality and policy-makers. Misinformation is an important emerging topic
in information science and sociology \citep{Metaxa-Kakavouli2017}. In essence,
this is a study of modern epistemology. How do we come to know what we know? How
do we come to be misinformed? If our news media and other sources of factual
information are pushing users in biased ways and ways that depend on the
demographics of the user, then we cannot be assured that this information is
reliable. Understanding how this differs by demographic variables has important
implications for social research and policy.






%this is all to turn tale horizontal and clean page
\newpage
\newgeometry{margin=1in} % modify this if you need even more space
\begin{landscape}
\thispagestyle{empty}

% Predicting COVID Knowledge: Logistic Regressions Estimating Correct COVID Knowledge Conditional on Demographic and Media Predictors.

\begin{table}
  \caption[Predicting COVID Knowledge: Logistic Regressions Estimating Correct COVID Knowledge Conditional on Demographic and Media Predictors.]
  {\emph{Predicting COVID Knowledge: Logistic Regressions Estimating Correct
  COVID Knowledge Conditional on Demographic and Media Predictors. Odds ratios
  (and linearized standard errors) for each knowledge question. }}
  \label{table:COVIDKnowledgeORs}
%  \small
  \begin{tabular}{l|c|c|c|c}
    \hline   % adds horizontal lines to the top of the table
                          & Anthony Fauci            & Antibody tests          & Unemployment      & Some states\\
                          & is an infectious disease & detect previous         & rate around 15\%  & had no stay-at- \\
    Variable              & expert \& govt. adviser  & coronavirus infections  & in June           & home order \\
    \hline
    \multicolumn{4}{l}{{\bf News Source} (Comparison:  International news outlets)}   \\
      \enspace National news outlets &  1.66 (0.31)***    & 1.68 (0.25)**      &  1.22 (0.16)      & 1.38 (0.19)*     \\
      \enspace Local news outlets    &  0.45 (0.08)***    & 0.69 (0.11)*       &  0.69 (0.10)*     & 0.80 (0.11)      \\
      \enspace Trump or coronavirus  &  0.86 (0.17)       & 0.96 (0.15)        &  0.91 (0.13)      & 1.11 (0.16)     \\
      \enspace task force            &                    &                    &                   &      \\
      \enspace Biden campaign        &  0.10 (0.08)***    & 0.27 (0.17)*       &  0.36 (0.28)      & 0.81 (0.60)   \\
      \enspace State and local       &  0.92 (0.19)       & 1.41 (0.24)*       &  0.75 (0.11)      & 1.17 (0.18)  \\
      \enspace elected officials     &                    &                    &                   &       \\
      \enspace Public health         &  0.77 (0.14)       & 1.36 (0.21)*       &  0.89 (0.12)      & 1.19       \\
      \enspace orgs. \& officials    &                    &                    &                   &       \\
      \enspace Friends, family       &  0.13 (0.03)***    & 0.39 (0.09)***     &  0.35 (0.08)***   & 0.48 (0.10)***   \\
      \enspace \& neighbors          &                    &                    &                   &       \\
      \enspace Community or          &  0.14 (0.06)***    & 0.44 (0.20)        &  0.80 (0.39)      & 0.70 (0.36)         \\
      \enspace neighborhood news     &                    &                    &                   &       \\
      \enspace Online forums         &  0.41 (0.10)***    & 0.80 (0.17)        &  0.60 (0.13)*     & 1.09 (0.23)        \\
      \enspace or discussion groups  &                    &                    &                   &      \\
    \multicolumn{5}{l}{{\bf Race} (comparison: White)}   \\
      \enspace Black                 &  0.46 (0.05)***    & 0.29 (0.03)***     &  0.74 (0.08)**    & 0.52 (0.05)***   \\
      \enspace Asian                 &  0.82 (0.18)       & 0.44 (0.08)***     &  0.93 (0.14)      & 0.81 (0.13)          \\
      \enspace Hispanic              &  0.50 (0.04)***    & 0.44 (0.03)***     &  0.72 (0.05)***   & 0.62 (0.04)***    \\
      \enspace Other/Mixed           &  0.55 (0.10)**     & 0.58 (0.09)**      &  0.66 (0.10)**    & 0.73 (0.11)*       \\
    \multicolumn{5}{l}{{\bf Income} (comparison: \$0 to \$10,000)}   \\
      \enspace \$10,000 to \$20,000  & 1.35 (0.24)        & 1.30 (0.22)        &  1.06 (0.19)      & 1.06 (0.17)         \\
      \enspace \$20,000 to \$30,000  & 1.83 (0.31)***     & 1.98 (0.32)***     &  1.52 (0.27)*     & 1.17 (0.18)          \\
      \enspace \$30,000 to \$40,000  & 2.12 (0.36)***     & 2.19 (0.36)***     &  2.16 (0.27)***   & 1.31 (0.21)           \\
      \enspace \$40,000 to \$50,000  & 2.84 (0.48)***     & 2.48 (0.40)***     &  1.76 (0.30)**    & 1.40 (0.21)*            \\
      \enspace \$50,000 to \$75,000  & 4.21 (0.66)***     & 3.24 (0.49)***     &  2.45 (0.39)***   & 2.00 (0.28)***            \\
      \enspace \$75,000 to \$100,000 & 5.79 (0.98)***     & 4.83 (0.75)***     &  2.72 (0.43)***   & 2.00 (0.29)***            \\
      \enspace \$100,000 to \$150,000 & 7.51 (1.31)***    & 5.97 (0.94)***     &  2.96 (0.47)***   & 2.23 (0.32)***    \\
      \enspace \$150,000 and above  & 10.57 (2.12)***     & 8.92 (1.52)***     &  4.08 (0.66)***   & 2.20 (0.32)**            \\
    \hline  % adds horizontal line to the bottom edges
    N                     & 9173                     &  9173                   &   9173            &  9173 \\
    \hline
  \end{tabular}
\end{table}

\end{landscape}
\restoregeometry
\newpage


%------------------------------------------------------------------------------
\newpage
\hypertarget{references}{%
\label{references}}
\renewcommand{\bibname}{References}

\bibliographystyle{/Users/mollymking/Documents/SocResearch/LatexStyleFiles/asa} %tells to use asr style file instead of natbib
\bibliography{/Users/mollymking/Documents/SocResearch/Mendeley/library}

%----------------------------------------------------------------------------------------

\end{document}
